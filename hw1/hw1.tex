%%%%%%%%%%%%%%%%%%%%%%%%%%%%%%%%%%%%%%%%%
% Structured General Purpose Assignment
% LaTeX Template
%
% This template has been downloaded from:
% http://www.latextemplates.com
%
% Original author:
% Ted Pavlic (http://www.tedpavlic.com)
%
% Note:
% The \lipsum[#] commands throughout this template generate dummy text
% to fill the template out. These commands should all be removed when 
% writing assignment content.
%
%%%%%%%%%%%%%%%%%%%%%%%%%%%%%%%%%%%%%%%%%

%----------------------------------------------------------------------------------------
%	PACKAGES AND OTHER DOCUMENT CONFIGURATIONS
%----------------------------------------------------------------------------------------

\documentclass{article}

\usepackage{fancyhdr} % Required for custom headers
\usepackage{lastpage} % Required to determine the last page for the footer
\usepackage{extramarks} % Required for headers and footers
\usepackage{graphicx} % Required to insert images
\usepackage{lipsum} % Used for inserting dummy 'Lorem ipsum' text into the template
\usepackage{listings}
\usepackage{color}
\usepackage{amsmath}

\definecolor{dkgreen}{rgb}{0,0.6,0}
\definecolor{gray}{rgb}{0.5,0.5,0.5}
\definecolor{mauve}{rgb}{0.58,0,0.82}

\lstset{frame=tb,
  language=C++,
  aboveskip=3mm,
  belowskip=3mm,
  showstringspaces=false,
  columns=flexible,
  basicstyle={\small\ttfamily},
  numbers=none,
  numberstyle=\tiny\color{gray},
  keywordstyle=\color{blue},
  commentstyle=\color{dkgreen},
  stringstyle=\color{mauve},
  breaklines=true,
  breakatwhitespace=true
  tabsize=3
}

% Margins
\topmargin=-0.45in
\evensidemargin=0in
\oddsidemargin=0in
\textwidth=6.5in
\textheight=9.0in
\headsep=0.25in 

\linespread{1.1} % Line spacing

% Set up the header and footer
\pagestyle{fancy}
\lhead{\hmwkAuthorName} % Top left header
\chead{\hmwkClass\ (\hmwkClassInstructor\ \hmwkClassTime): \hmwkTitle} % Top center header
\rhead{\firstxmark} % Top right header
\lfoot{\lastxmark} % Bottom left footer
\cfoot{} % Bottom center footer
\rfoot{Page\ \thepage\ of\ \pageref{LastPage}} % Bottom right footer
\renewcommand\headrulewidth{0.4pt} % Size of the header rule
\renewcommand\footrulewidth{0.4pt} % Size of the footer rule

\setlength\parindent{0pt} % Removes all indentation from paragraphs

%----------------------------------------------------------------------------------------
%	DOCUMENT STRUCTURE COMMANDS
%	Skip this unless you know what you're doing
%----------------------------------------------------------------------------------------

% Header and footer for when a page split occurs within a problem environment
\newcommand{\enterProblemHeader}[1]{
\nobreak\extramarks{#1}{#1 continued on next page\ldots}\nobreak
\nobreak\extramarks{#1 (continued)}{#1 continued on next page\ldots}\nobreak
}

% Header and footer for when a page split occurs between problem environments
\newcommand{\exitProblemHeader}[1]{
\nobreak\extramarks{#1 (continued)}{#1 continued on next page\ldots}\nobreak
\nobreak\extramarks{#1}{}\nobreak
}

\setcounter{secnumdepth}{0} % Removes default section numbers
\newcounter{homeworkProblemCounter} % Creates a counter to keep track of the number of problems

\newcommand{\homeworkProblemName}{}
\newenvironment{homeworkProblem}[1][Problem \arabic{homeworkProblemCounter}]{ % Makes a new environment called homeworkProblem which takes 1 argument (custom name) but the default is "Problem #"
\stepcounter{homeworkProblemCounter} % Increase counter for number of problems
\renewcommand{\homeworkProblemName}{#1} % Assign \homeworkProblemName the name of the problem
\section{\homeworkProblemName} % Make a section in the document with the custom problem count
\enterProblemHeader{\homeworkProblemName} % Header and footer within the environment
}{
\exitProblemHeader{\homeworkProblemName} % Header and footer after the environment
}

\newcommand{\problemAnswer}[1]{ % Defines the problem answer command with the content as the only argument
\noindent\framebox[\columnwidth][c]{\begin{minipage}{0.98\columnwidth}#1\end{minipage}} % Makes the box around the problem answer and puts the content inside
}

\newcommand{\homeworkSectionName}{}
\newenvironment{homeworkSection}[1]{ % New environment for sections within homework problems, takes 1 argument - the name of the section
\renewcommand{\homeworkSectionName}{#1} % Assign \homeworkSectionName to the name of the section from the environment argument
\subsection{\homeworkSectionName} % Make a subsection with the custom name of the subsection
\enterProblemHeader{\homeworkProblemName\ [\homeworkSectionName]} % Header and footer within the environment
}{
\enterProblemHeader{\homeworkProblemName} % Header and footer after the environment
}
   
%----------------------------------------------------------------------------------------
%	NAME AND CLASS SECTION
%----------------------------------------------------------------------------------------

\newcommand{\hmwkTitle}{Homework 1} % Assignment title
\newcommand{\hmwkDueDate}{Sep 11,\ 2014} % Due date
\newcommand{\hmwkClass}{MTH 9821} % Course/class
\newcommand{\hmwkClassTime}{Weiyi Chen, Xia Hua, Sam Pfeiffer, Xiaoyu Zhang} % Class/lecture time
\newcommand{\hmwkClassInstructor}{} % Teacher/lecturer
\newcommand{\hmwkAuthorName}{} % Your name

%----------------------------------------------------------------------------------------
%	TITLE PAGE
%----------------------------------------------------------------------------------------

\title{
\vspace{2in}
\textmd{\textbf{\hmwkClass:\ \hmwkTitle}}\\
\normalsize\vspace{0.1in}\small{Due\ on\ \hmwkDueDate}\\
\vspace{0.1in}\large{\textit{\hmwkClassInstructor\ \hmwkClassTime}}
\vspace{3in}
}

\author{\textbf{\hmwkAuthorName}}
\date{} % Insert date here if you want it to appear below your name

%----------------------------------------------------------------------------------------

\begin{document}

\maketitle

%----------------------------------------------------------------------------------------
%	TABLE OF CONTENTS
%----------------------------------------------------------------------------------------

%\setcounter{tocdepth}{1} % Uncomment this line if you don't want subsections listed in the ToC

%\newpage
%\tableofcontents
\textbf{Weiyi Chen:} \#4, 5, 6, 7, 8 \\
\textbf{Xia Hua:} \#1, 2, 6, 7, 8 \\
\textbf{Sam Pfeiffer:} \#3, 4, 5, 7, 8 \\
\textbf{Xiaoyu Zhang:} \#1, 2, 3, 7, 8 

\newpage

%----------------------------------------------------------------------------------------
%   PROBLEM 1
%----------------------------------------------------------------------------------------

\begin{homeworkProblem}
    Since $L_2$ and $U_1$ are nonsingular, they have inverses. This allows us to say,
    \begin{align}
        L_1U_1 &= L_2U_2 \\
        L_2^{-1}L_1U_1U_1^{-1} &= L_2^{-1}L_2U_2U_1^{-1} \\
        L_2^{-1}L_1 &= U_2U_1^{-1}
    \end{align}
    From the properties of triangular matrices, $U_1U_1^{-1}$ is upper triangular and $L_2^{-1}L_1$ is lower triagular. For them to be equal it must be that $U_2U_1^{-1}$ and $L_2^{-1}L_1$ are diagonal matrices. Define the diagonal matrix
    \begin{equation}
        D = U_2U_1^{-1} = L_2^{-1}L_1
    \end{equation}
    then we have
    \begin{align}
        D = U_2U_1^{-1} &\Rightarrow U_2=DU_1 \\
        D = L_2^{-1}L_1 &\Rightarrow L_2=L_1D^{-1}
    \end{align}
\end{homeworkProblem}

%----------------------------------------------------------------------------------------
%   PROBLEM 2
%----------------------------------------------------------------------------------------

\begin{homeworkProblem}
    Let's say the given matrix is $A$. Apply the Pseudocde for LU decomposition without pivoting (Table 2.5 on textbook) to $A$, i.e.,
    \begin{lstlisting}
        [L,U] = lu_no_pivoting(A)
    \end{lstlisting}
    We generate the output as, the lower triangular matrix with entries $1$ on main diagonal,
    \begin{equation}
        L = \left(\begin{array}{ccccc}
        1 & 0 & 0 & 0 & 0 \\
        -1 & 1 & 0 & 0 & 0 \\
        -1 & -1 & 1 & 0 & 0 \\
        -1 & -1 & -1 & 1 & 0 \\
        -1 & -1 & -1 & -1 & 1
        \end{array} \right)
    \end{equation}
    and the upper triangular matrix
    \begin{equation}
        U = \left(\begin{array}{ccccc}
        1 & 0 & 0 & 0 & 1 \\
        0 & 1 & 0 & 0 & 2 \\
        0 & 0 & 1 & 0 & 4 \\
        0 & 0 & 0 & 1 & 8 \\
        0 & 0 & 0 & 0 & 16
        \end{array} \right)
    \end{equation}
    such that $A=LU$.
\end{homeworkProblem}

%----------------------------------------------------------------------------------------
%   PROBLEM 3
%----------------------------------------------------------------------------------------

\begin{homeworkProblem}
    \begin{homeworkSection}{(i)}
        The $2\times2$ leading principle minor of $A$ is
        \begin{equation}
            det\left(\begin{array}{cc}2&-1\\-1&1\end{array}\right) = 2 \times 1 - (-2) \times (-1) = 0
        \end{equation}
    \end{homeworkSection}
    \begin{homeworkSection}{(ii)}
        Let $L$ and $U$ be the LU factors of $A$. The entries of the first row of $U$ are given $U(1,k)=A(1,k)$ for $k=1:3$:
        \begin{equation}
            U(1,1) = 2, U(1,2) = -1, U(1,3) = 1
        \end{equation}
        The entries of the first column of $L$ are given by $L(k,1)=A(k,1)/U(1,1)$ for $k=1:3$:
        \begin{equation}
            L(1,1)=1, L(2,1)=\frac{-2}{U(1,1)}=-1, L(3,1)=\frac{4}{U(1,1)}=2
        \end{equation}
        The current forms of $L$ and $U$ are
        \begin{equation}
            L = \left(\begin{array}{ccc}
            1 & 0 & 0 \\
            -1 & 1 & 0 \\
            2 & L(3,2) & 1 
            \end{array} \right),
            U = \left(\begin{array}{ccc}
            2 & -1 & 1 \\
            0 & U(2,2) & U(2,3) \\
            0 & 0 & U(3,3)
            \end{array} \right)            
        \end{equation}
        The updated form of the $2\times2$ matrix $A(3:4,3:4)$ is
        \begin{align}
            A(2:3,2:3) &= A(2:3,2:3) - L(2:3,1)U(1,2:3) \\
            &= \left(\begin{array}{cc} 1 & 3 \\ 0 & -1 \end{array}\right) - \left(\begin{array}{c} -1 \\ 2 \end{array}\right)(-1,1) \\
            &= \left(\begin{array}{cc} 0 & 4 \\ 2 & -3 \end{array}\right)
        \end{align}
        The unknown entries from the second row of $U$ and from the second column of $L$ can be computed from the $2\times2$ matrix above in the same way, the entries of the first row of $U(2:3)$ are given $U(2,k)=A(2,k)$ for $k=2:3$:
        \begin{equation}
            U(2,2) = 0, U(2,3) = 4
        \end{equation}
        The entries of the first column of $L$ are given by $L(k,2)=A(k,2)/U(2,2)$ for $k=2:3$:
        \begin{equation}
            L(3,2) = \frac{A(3,2)}{U(2,2)}
        \end{equation}
        However $U(2,2)=0$, the division by $U(2,2)$ cannot be performed when trying to compute this second row of $L$.
    \end{homeworkSection}
    \begin{homeworkSection}{(iii)}
        Since
        \begin{equation}
            det(A) = -16
        \end{equation}
        then $A$ is nonsingular, we just apply the Pseudocde for LU decomposition with row pivoting (Table 2.10 on textbook) to $A$, i.e.,
        \begin{lstlisting}
            [P,L,U] = lu_row_pivoting(A)
        \end{lstlisting}
        We generate the output as, the permutation matrix,
        \begin{equation}
            P = \left(\begin{array}{ccc}
                0 & 0 & 1 \\
                0 & 1 & 0 \\
                1 & 0 & 0 
                \end{array} \right)
        \end{equation}
        the lower triangular matrix with entries $1$ on main diagonal,
        \begin{equation}
            L = \left(\begin{array}{ccc}
                1 & 0 & 0 \\
                -0.5 & 1 & 0 \\
                0.5 & -1 & 1 
                \end{array} \right)            
        \end{equation}
        and the upper triangular matrix
        \begin{equation}
            U = \left(\begin{array}{ccc}
                4 & 0 & -1 \\
                0 & 1 & 2.5 \\
                0 & 0 & 4
                \end{array} \right)  
        \end{equation}
        such that $PA=LU$.
    \end{homeworkSection}
\end{homeworkProblem}

%----------------------------------------------------------------------------------------
%   PROBLEM 4
%----------------------------------------------------------------------------------------

\begin{homeworkProblem}
    Pseudocode for the forward substitution corresponding to a lower triangular banded matrix of band $m$
    \begin{lstlisting}
        Function Call:
        x = forward_subst_band(L,b,m)

        Input:
        L = nonsingular banded lower triangular matrix of size n and band m
        b = column vector of size n
        m = band of banded matrix

        Output:
        x = solution to Lx=b

        x[1] = b[1] / L[1,1]
        for j = 2:n
            sum = 0
            for k = max{1,(j-m)}:(j-1)
                sum = sum + L[j,k]x[k]
            end
            x[j] = (b[j]-sum)/L[j,j]
        end
    \end{lstlisting}
    Computing $x(1)$ requires $1$ operation. At step $j=2:(m+1)$, 
    \begin{equation}
        \max\{1,(j-m)\} = 1
    \end{equation}
    the "for" loop to compute the term sum requires $2(j-1)$ operations. Thus, the "for" loop to compute $x(j)$ for $j=2:(m+1)$ requires $2(j-1)+2=2j$ operations.
    At step $j=(m+2):n$,
    \begin{equation}
         \max\{1,(j-m)\} = j-m
    \end{equation} 
    the "for" loop to compute the term sum requires $2m$ operations. Thus, the "for" loop to compute $x(j)$ for $j=(m+2:n)$ requires $2m+2$ operations. \\
    Then the total number of operations required by the Forward Substitution is
    \begin{align}
        1 + \sum_{j=2}^{m+1} {2j} + \sum_{j=m+2}^n {2m+2} &= 1 + m(m+3) + (n-m-1)(2m+2) \\
        &= (2m+2)n -m^2 -m -1 \\
        &= 2mn - m^2 + O(n)
    \end{align}
\end{homeworkProblem}

%----------------------------------------------------------------------------------------
%   PROBLEM 5
%----------------------------------------------------------------------------------------

\begin{homeworkProblem}
    (Symmetrical to )Pseudocode for the backward substitution corresponding to an upper triangular banded matrix of band $m$
    \begin{lstlisting}
        Function Call:
        x = backward_subst_band(U,b)

        Input:
        U = nonsingular banded upper triangular matrix of size n and band m
        b = column vector of size n

        Output:
        x = solution to Ux=b

        x[n] = b[n] / U[n,n]
        for j = (n-1):1
            sum = 0
            for k = (j+1):min{n,(j+m)}
                sum = sum + U[j,k]x[k]
            end
            x[j] = (b[j]-sum)/U[j,j]
        end
    \end{lstlisting}
    Computing $x(n)$ requires $1$ operation. At step $j=(n-1):(n-m-1)$, the "for" loop to compute the term sum requires $2(n-j)$ operations. Thus, the "for" loop to compute $x(j)$ for $j=(n-1):(n-m-1)$ requires $2(n-j)+2=2n-2j+2$ operations. At step $j=(n-m-2):1$, the "for" loop to compute the term sum requires $2m$ operations. Thus, the "for" loop to compute $x(j)$ for $j=(n-m-2:1)$ requires $2m+2$ operations. \\
    Then the total number of operations required by the Forward Substitution is
    \begin{align}
        1 + \sum_{j=n-m-1}^{n-1} {2n-2j+2} + \sum_{1}^{n-m-2} {2m+2} &= 1 + m(m+3) + (n-m-1)(2m+2) \\
        &= (2m+2)n -m^2 -m -1 \\
        &= 2mn - m^2 + O(n) 
    \end{align}
\end{homeworkProblem}

%----------------------------------------------------------------------------------------
%   PROBLEM 6
%----------------------------------------------------------------------------------------

\begin{homeworkProblem}
    Given the fact that described in the problem, the pseudocode for the LU composition without pivoting for banded matrices of band $m \ge 1$ is
    \begin{lstlisting}
        Function Call:
        [L,U] = lu_no_pivoting_band(A)

        Input:
        A = nonsingular banded matrix of size n and band m with LU decomposition

        Output:
        L = lower banded triangular matrix with entries 1 on main diagonal and band m
        U = uppder banded triangular matrix with band m
        such that A = LU

        Initialize L = 0, U = 0 
        for i = 1:(n-1)
            for k = i:min{(i+m),n}
                U(i,k) = A(i,k)
                L(k,i) = A(k,i)/U(i,i)
            end
            for j = (i+1):min{(i+m),n}
                for k = (i+1):min{(i+m),n}
                    A(j,k) = A(j,k) - L(j,i)U(i,k)
                end
            end
        end
        L(n,n) = 1, U(n,n) = A(n,n)
    \end{lstlisting}
    At step $i = 1:(n-m-1)$, the "for" loop to compute the i-th row of $U$ and the i-th column of $L$ requires $m+1$ operations. The double "for" loop to update $A(i+1:i+m,i+1:i+m)$ requires
    \begin{equation}
        \sum_{i+1}^{i+m} \sum_{i+1}^{i+m} 2 = 2m^2
    \end{equation}
    operations. Then accounting for the outside "for" loop for $i=1:(n-m-1)$, we obtain that the operation count for the LU decomposition without pivoting is
    \begin{equation}
        \sum_{i=1}^{n-m-1} (2m^2+m+1) = (n-m-1)(2m^2+m+1)
    \end{equation} 
    At step $i = (n-m):(n-1)$, the "for" loop to compute the i-th row of $U$ and the i-th column of $L$ requires $n-i+1$ operations. The double "for" loop to update $A(i+1:n,i+1:n)$ requires
    \begin{equation}
        \sum_{j=i+1}^{n} \sum_{k=i+1}^{n} 2 = 2(n-i)^2
    \end{equation}
    operations. Then accounting for the outside "for" loop for $i=(n-m):(n-1)$, we obtain that the operation count for the LU decomposition without pivoting is
    \begin{align}
        \sum_{i=n-m}^{n-1} [2(n-i)^2+(n-i+1)] = \frac{2}{3}m^3 + \frac{3}{2}m^2 + \frac{11}{6}m
    \end{align}
    Therefore the total operation count is
    \begin{align}
        &~~\sum_{i=1}^{n-m-1} (2m^2+m+1) + \sum_{i=n-m}^{n-1} [2(n-i)^2+(n-i+1)] \\
        &= (2m^2+m+1)n-\frac{1}{6}m(m+1)(8m+1)\\
        &= 2m^2n - \frac{4}{3}m^3 + O(mn)
    \end{align}
\end{homeworkProblem}


%----------------------------------------------------------------------------------------
%   PROBLEM 7
%----------------------------------------------------------------------------------------

\begin{homeworkProblem}
    C++ codes for problem 7 and 8:
    \begin{lstlisting}
#ifndef LU_DECOMPOSITION_HPP_
#define LU_DECOMPOSITION_HPP_

#include <Eigen/Dense>
#include <boost/tuple/tuple.hpp>

using namespace Eigen;

VectorXd forward_subst(MatrixXd L, VectorXd b){
    /*
    @summary: Forward substitution
    @param L: nonsingular lower triangular matrix of size n
    @param b: column vector of size n
    @return x: solution to Lx=b 
    */
    int n = sqrt(L.size());
    VectorXd x(n);
    x(0) = b(0) / L(0,0);
    for (int j = 1; j < n; ++j){
        double sum = 0;
        for (int k = 0; k < j; ++k)
            sum += L(j,k) * x(k);
        x(j) = (b(j) - sum) / L(j,j);
    }
    return x;
}

VectorXd backward_subst(MatrixXd U, VectorXd b){
    /*
    @summary: Backward substitution
    @param U: nonsingular upper triangular matrix of size n
    @param b: column vector of size n
    @return x: solution to Ux=b 
    */
    int n = sqrt(U.size());
    VectorXd x(n);
    x(n-1) = b(n-1) / U(n-1,n-1);
    for (int j = n-2; j >= 0; --j){
        double sum = 0.0;
        for (int k = j+1; k < n; ++k)
            sum += U(j,k) * x(k);
        x(j) = (b(j) - sum) / U(j,j);
    }
    return x;
}

boost::tuple<MatrixXd, MatrixXd> lu_no_pivoting(MatrixXd A) {
    /*
    @summary: LU decomposition without pivoting
    @param A: nonsingular matrix of size n with LU decomposition
    @return L: lower triangular matrix with entries 1 on main diagonal
    @return U: upper triangular matrix such that A=LU 
    */
    int n = sqrt(A.size());
    MatrixXd L(n,n), U(n,n);
    L.setZero();
    U.setZero();
    for (int i = 0; i < n-1; ++i) {
        for (int k = i; k < n; ++k) {
            U(i,k) = A(i,k);
            L(k,i) = A(k,i) / U(i,i);
        }
        A.block(i+1,i+1,n-i-1,n-i-1) -= L.block(i+1,i,n-i-1,1)*U.block(i,i+1,1,n-i-1);
    }
    L(n-1,n-1) = 1;
    U(n-1,n-1) = A(n-1, n-1);
    return boost::make_tuple(L, U);
}

VectorXd linear_solve_LU_no_pivoting(MatrixXd A, VectorXd b) {
    /*
    @summary: linear solver using LU decomposition without pivoting
    @param A: nonsingular square matrix of size n with LU decomposition
    @param b: column vector of size n
    @return x: solution to Ax=b
    */
    int n = sqrt(A.size());
    MatrixXd L(n,n), U(n,n);
    boost::tie(L, U) = lu_no_pivoting(A);
    VectorXd y(n), x(n);
    y = forward_subst(L, b);
    x = backward_subst(U, y);
    return x;
}

boost::tuple<VectorXi, MatrixXd, MatrixXd> lu_row_pivoting(MatrixXd A) {
    /*
    @summary: LU decomposition with row pivoting
    @param A: nonsingular matrix of size n
    @return P: permutation matrix, stored as vector of its diagonal entries
    @return L: lower triangular matrix with entries 1 on main diagonal
    @return U: upper triangular matrix such that PA=LU
    */
    int n = sqrt(A.size());
    VectorXi P = VectorXi::LinSpaced(n,1,n);    // initialize P as an identity matrix
    MatrixXd L(n,n), U(n,n);
    L.setIdentity();                            // initialize L as an identity matrix
    for (int i = 0; i < n; ++i){
        ArrayXd::Index i_row, i_column;
        A.block(i,i,n-i,1).array().abs().maxCoeff(&i_row, &i_column);
        ArrayXd::Index i_max = i_row + i;       // find i_max, index of largest entry in absolute value from vector A(i:n,i)
        A.row(i).swap(A.row(i_max));            // switch rows i and i_max of A
        P.row(i_max).swap(P.row(i));                    // update matrix P
        if (i > 0)
            L.block(i,0,1,i).swap(L.block(i_max,0,1,i));        // switch rows i and i_max of L
        for (int j = i; j < n; j++) {
            L(j,i) = A(j,i) / A(i,i);           // compute column i of L
            U(i,j) = A(i,j);                    // compute row i of U
        }
        A.block(i+1,i+1,n-i-1,n-i-1) -= L.block(i+1,i,n-i-1,1)*U.block(i,i+1,1,n-i-1);
    }
    L(n-1,n-1) = 1;
    U(n-1,n-1) = A(n-1,n-1);
    return boost::make_tuple(P,L,U);
}

#endif /* LU_DECOMPOSITION_HPP_ */
    \end{lstlisting}
\end{homeworkProblem}

\end{document}